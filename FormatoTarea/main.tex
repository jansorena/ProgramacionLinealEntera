\documentclass[11pt,addpoints]{article}
\usepackage[T1]{fontenc}
\usepackage[utf8]{inputenc}
\usepackage[spanish]{babel}
\usepackage{graphicx}
\parindent = 0cm % sangria distancia
%%
\usepackage{amsmath}
\usepackage{amssymb,amsfonts,latexsym,cancel}
\providecommand{\abs}[1]{\lvert#1\rvert}
\providecommand{\norm}[1]{\lVert#1\rVert}
\usepackage{multirow}
%%
\usepackage[table,xcdraw]{xcolor}
\usepackage[lmargin=2cm,rmargin=2cm,top=2cm,bottom=2.5cm]{geometry}
\usepackage{xspace}
\usepackage{fancyhdr}
\pagestyle{fancy}
\fancyhead{}
\fancyhead[L]{FACULTAD DE INGENIERÍA\\DEPTO DE INGENIERÍA INDUSTRIAL \\ OPTIMIZACION I (546351)}
\fancyhead[R]{\includegraphics[scale=0.13]{escudoudec}}
\fancyfoot{\thepage}
\fancyfoot[R]{Semestre 2023-2}
\fancyfoot[L]{RM/MBZ/MBS}
\renewcommand{\headrulewidth}{0.9pt}
\renewcommand{\footrulewidth}{0.5pt}
\vspace*{0.1cm}
%%%%%%%%%%%%%%%%%%%%%%%%%%%%%%%%%%%%%%%%%%%%%%%%%%%%%%%%%%%%%%%%%%%%%%%%%%%%%%%%%%%%%%%
\begin{document}

\begin{center}

{\Large \bfseries Tarea Computacional 1: Modelación matemática y Solver de $\text{Excel}^\circledR$}\\[0.5cm]	%%%
%\HRule \\[0.2cm] 
\begin{minipage}{1.0\textwidth} 
\begin{center}	
\textbf{Integrantes}\\%%%
\textsc{Barrientos Zagal Matías Sebastián - 2000667711\\
Barrientos Zagal Matías Sebastián - 2000667711
\\[0.4cm]
}

\end{center}

\end{minipage}\\[0.2cm]
\end{center}

%%%%%%%%%%%%%% COMIENZAN LOS EJERCICIOS %%%%%%%%%%%%%%%%%%%%%%%%%
\section{Situación Propuesta}
Escriban el problema que seleccionaron con sus respectivos supuestos y datos necesarios para plantear el modelo matemático.
\section{Variables de decisión}
En este modelo, se consideraron tres variables de decisión que son binarias y se definen a continuación:\\
\begin{equation*} 
	x_{ij} =\left\lbrace 
    \begin{array}{ll} 1\ \textup{ \ ,\ }  \text{si el arco}\ (i,j) \ \text{es parte de la ruta} & \\ 0 \textup{\ , \ } \text{e.o.c} \end{array} \right.
    \end{equation*}\\
    \begin{equation*} 
	y_{ik} =\left\lbrace 
    \begin{array}{ll} 1\ \textup{ \ ,\ }  \text{si la instalación}\ i \ \text{cubre al cliente } k \ & \\ 0 \textup{\ , \ } \text{e.o.c} \end{array} \right.
    \end{equation*}\\
    \begin{equation*} 
	z_{i} =\left\lbrace 
    \begin{array}{ll} 1\ \textup{ \ ,\ }  \text{si la instalación}\ i \ \text{es parte de la ruta} & \\ 0 \textup{\ , \ } \text{e.o.c} \end{array} \right.\end{equation*}\\
    \begin{equation*}
    g_{ij} = \text{flujo que pasa por el arco (i,j)}    
    \end{equation*}
$\forall i \in N=0,...,n$, $\forall j \in N=0,...,n$ y $\forall k \in C=1,...,m$. Con $n$ el número de instalaciones, donde el depósito es $i=0$ y donde $m$ es el número de clientes respectivos. 
\section{Parámetros del Modelo}
Los parámetros del modelo son:
\begin{align*}
    N &= \text{Conjunto de nodos}\\
    A &= \text{Conjunto de arcos}\\
    0 &= \text{nodo de origen y término (depot)}\\
    t_{ij} &= \text{Distancia entre}\ i \ \text{hasta}\ j\\
    P &= \text{Total beneficio a recolectar}\\
    M &= \text{Número real muy grande}\\
I_c &= \text{Conjunto de instalaciones capaces de cubrir a los clientes}\ c \in C\\
    p_i &= \text{beneficio de cubrir al cliente }i
\end{align*}
\newpage %No borrar!!
\vspace*{0.3cm} %No borrar!!
\section{Función objetivo}
Explicar brevemente la función objetivo.
\begin{gather*}
    \text{ Minimizar}\ \ Z = \sum_{i=0}^{n} \sum_{j=1}^{n} t_{ij}x_{ij} - \sum_{(i,j)\in A : i \in I_c } y_{ij} - \sum_{i \in N} z_i\\
\end{gather*}

\section{Restricciones}
Explicar brevemente cada restricción.
\begin{gather*}
    \sum_{j \in N} x_{ij}=1 \ \ \ \ (\forall i \in N\setminus\{0\}) \ \ \ \ (1)\\
    \sum_{i \in N} x_{ij} = 1 \ \ \ \ (\forall j \in N\setminus\{0\}) \ \ \ \ (2) \\
     \sum_{j \in N} x_{0j} = 1 \ \ \ \ \ \ \ \  \ \ \ \   \ \ \ \  \ \ \ \  \ \ \ \ \ \ \ (3) \\
    \sum_{i \in N} x_{i0} = 1 \ \ \ \ \ \ \ \  \ \ \ \   \ \ \ \  \ \ \ \  \ \ \ \ \ \ \ \ (4) \\
     \sum_{(0,j) \in A} g_{0j} = \sum_{(i,j)\in A\setminus{0}:i \in I_{c}} y_{ij} + \sum_{i \in N: i \neq 0} z_i \ \ \ \ \ \ (5) \\
     \sum_{(i,j) \in A} g_{ij} - \sum_{(i,j) \in A} g_{ji} = z_i + \sum_{i \in I_c :(i,j)\in A} y_{ij} \ \ \ \  (\forall i \in N: i\neq0, j\neq0) \ \ \ \ \ (6) \\
     g_{ij} \leq Mx_{ij} \ \ \ \ (\forall (i,j) \in A: i \notin I_c) \ \ \ \ \ \ \ (7)\\
     g_{ij} \leq Mx_{ij} + z_{ij} \ \ \ \ (\forall (i,j) \in A: i \in I_c) \ \ \ \ \ (8)\\
     g_{ij} \leq Px_{ij} \ \ \ \ (\forall (0,j) \in A) \ \ \ \ \ \ \ \ \ (9)\\
    \text{\ \ \ \ } x_{ij} \in \{0,1\} \ , \ y_{ij} \in \{0,1\} \ , \ z_i \in \{0,1\} \ , \ g_{ij} \geq 0 \ \ \ \ \ (10)
\end{gather*}

\newpage %No borrar!!
\vspace*{0.3cm} %No borrar!!
\section{Modelo matemático}
El modelo de programación entera asociado a este problema, está dado por:\\
\begin{gather*}
    \text{ Minimizar}\ \ Z = \sum_{i=0}^{n} \sum_{j=1}^{n} t_{ij}x_{ij} - \sum_{(i,j)\in A : i \in I_c } y_{ij} - \sum_{i \in N} z_i\\
    \text{ sujeto a:}\\
     \sum_{j \in N} x_{ij}=1 \ \ \ \ (\forall i \in N\setminus\{0\}) \ \ \ \ (1)\\
    \sum_{i \in N} x_{ij} = 1 \ \ \ \ (\forall j \in N\setminus\{0\}) \ \ \ \ (2) \\
     \sum_{j \in N} x_{0j} = 1 \ \ \ \ \ \ \ \  \ \ \ \   \ \ \ \  \ \ \ \  \ \ \ \ \ \ \ (3) \\
    \sum_{i \in N} x_{i0} = 1 \ \ \ \ \ \ \ \  \ \ \ \   \ \ \ \  \ \ \ \  \ \ \ \ \ \ \ \ (4) \\
     \sum_{(0,j) \in A} g_{0j} = \sum_{(i,j)\in A\setminus{0}:i \in I_{c}} y_{ij} + \sum_{i \in N: i \neq 0} z_i \ \ \ \ \ \ (5) \\
     \sum_{(i,j) \in A} g_{ij} - \sum_{(i,j) \in A} g_{ji} = z_i + \sum_{i \in I_c :(i,j)\in A} y_{ij} \ \ \ \  (\forall i \in N: i\neq0, j\neq0) \ \ \ \ \ (6) \\
     g_{ij} \leq Mx_{ij} \ \ \ \ (\forall (i,j) \in A: i \notin I_c) \ \ \ \ \ \ \ (7)\\
     g_{ij} \leq Mx_{ij} + z_{ij} \ \ \ \ (\forall (i,j) \in A: i \in I_c) \ \ \ \ \ (8)\\
     g_{ij} \leq Px_{ij} \ \ \ \ (\forall (0,j) \in A) \ \ \ \ \ \ \ \ \ (9)\\
    \text{\ \ \ \ } x_{ij} \in \{0,1\} \ , \ y_{ij} \in \{0,1\} \ , \ z_i \in \{0,1\} \ , \ g_{ij} \geq 0 \ \ \ \ \ (10)
\end{gather*}
% \newpage  %%En caso de que necesiten otra página.
% \vspace*{0.3cm} %%No borrar!!
\section{Resultados}
Aquí presenten el informe de respuestas, que se obtiene con el Solver e identifiquen el valor de las variables de decisión y de la función objetivo. Analizar los demás informes haciéndose algunas preguntas como las del laboratorio 1.

\begin{table}[h]
\centering
\begin{tabular}{|l|c|}
\hline
\textbf{Variable de decisión}                                    & \multicolumn{1}{l|}{\textbf{Valor (unidad de medida}} \\ \hline
$x_{11}$              & 5.54                                     \\ \hline
$x_{12}$              & 10.12                                     \\ \hline
$x_{21}$              & 1.2                                     \\ \hline
$x_{22}$              & 1.2                                     \\ \hline
$y_{ik}$ & 4                                     \\ \hline
$z_i$                                   & 5                                     \\ \hline

\end{tabular}
\caption{Solución óptima del problema propuesto.}
\label{tab:my-table}
\end{table}
\end{document}